\section{Wstęp}

Celem utworzenia niniejszego dokumentu jest opisanie oraz zaprezentowanie realizacji projektu systemu informatycznego dla firmy \emph{Money Solutions sp.~z~o.o.} Wspomniana firma jest liderem w~branży doradztwa finansowego, skierowanego głównie do~klientów prywatnych. Z~racji na~rosnące zapotrzebowanie rynkowe na~usługi doradcze w~zakresie prowadzenia budżetu domowego, firma \emph{Money Solutions} zdecydowała się na~zamówienie systemu pozwalającego na~zarządzanie finansami domowymi, który będzie odpłatnie udostępniany jej klientom.\\

Zabieg taki pozwoli na~transfer niektórych obowiązków związanych ze~zbieraniem informacji nt.~budżetu domowego klientów do~nich samych. Dzięki temu każdy z~pracowników firmy \emph{Money Solutions} będzie mógł świadczyć swoje usługi większej liczbie klientów, jednocześnie zwiększając zyski firmy. Ponadto, wspomniany system będzie mógł być również oferowany jako osobna usługa, bez wspomnianego wcześniej doradztwa pracowników firmy.\\

Realizacja systemu została zlecona przez firmę \emph{Money Solutions~sp.~z~o.o.}, zwaną dalej Zleceniodawcą, firmie \emph{Four Developers~sp.~z~o.o.}, zwanej dalej Zleceniobiorcą, autorem tego dokumentu. System powinien spełnić wszystkie wymagania biznesowe postawione przed Zleceniobiorcą, które zostaną opisane w~dalszej części dokumentu. Zleceniobiorca zobowiązuje się do~wykonania wszystkich opisanych w~dokumencie części systemu, uruchomienie jej w~siedzibie firmy \emph{Money Solutions} oraz utrzymania systemu po wdrożeniu (z~czym wiąże się miesięczna opłata). 
