\section{Zakres realizacji}
Niniejszy rozdział jest poświęcony opisowi zakładanego sposobu realizacji projektu. Zostają tutaj zdefiniowane zespoły, ich zadania oraz role, a~także uściślany jest wstępny zarys technologiczny projektu.

\subsection{Realizacja projektu}
Projekt będzie realizowany w~siedzibie firmy \emph{Four Developers~sp.~z~o.o.} z~wykorzystaniem metodyki \emph{Scrum}. Zleceniodawca zobowiązał się do~oddelegowania jednego z~doświadczonych pracowników, który zna specyfikę projektu oraz wymagania, które stawia przed nim Zleceniodawca. Osoba ta~będzie uczestniczyła w~pracach zespołu jako \emph{Product Owner}. Pozostałych członków zespołu zapewni firma \emph{Four Developers}.

\subsection{Zarys technologiczny}
Aplikacja tworzona w~ramach projektu zostanie zrealizowana jako aplikacja z~dostępem z~poziomu przeglądarki internetowej. Dzięki temu, klienci firmy \emph{Money Solutions} będą mieli ułatwiony dostęp do~aplikacji po~wykupieniu do~niej dostępu. Równocześnie, zarządzanie użytkownikami oraz danymi będzie prostsze z~racji na scentralizowanie danych na~serwerze bazodanowym.\\

Ta konkretna aplikacja będzie realizowana w architekturze trójwarstwowej -- ponieważ jest to najczęściej wykorzystywana architektura, kiery rozważamy aplikacje z~dostępem poprzez przeglądarkę internetową. Wspomniana architektura jest przykładem architektury klient-serwer. Aplikacja jest podzielona na warstwy, co sprawia, że rozwój i~ewentualne modyfikacje są łatwiejsze. Architektura ta składa się z trzech warstw: warstwy danych, warstwy aplikacyjnej  (logiki biznesowej), oraz warstwy prezentacji. Opis każdej z~warstw znajduje się poniżej:

\begin{itemize}
   \item Warstwa prezentacji -- jedyna widoczna dla użytkownika -- zapewnia interfejs dla aplikacji. Ze względu na~założenie projektowe -- dostęp za~pośrednictwem przeglądarki internetowej -- stos technologiczny sprowadza się do HTML, CSS oraz JavaScript.
   \item Warstwa logiki biznesowej -- poziom odpowiedzialny za~komunikację między dwoma innymi warstwami -- danych oraz prezentacji.
   \item Warstwa danych -- używana do~tworzenia zapytań, przechowywania i~zarządzania danymi aplikacji. Składa się ona z~systemu zarządzania bazą danych oraz biblioteki, która zapewnia dedykowane API dostępu do~danych zawartych w~bazie danych z~poziomu kodu aplikacji.
\end{itemize}

\newpage
\subsection{Zespół}
W~skład zespołu będą wchodzili -- oprócz wspomnianego wcześniej \emph{Product Ownera} -- analitycy oraz projektanci.

\subsubsection{Analitycy -- wymagania}
Analitycy będą zajmować się prowadzeniem analizy biznesowej, badaniem potrzeb klientów, projektując rozwiązania dla systemu. Każdy z~analityków jest odpowiedzialny za~sprecyzowanie wymagań (zarówno funkcjonalnych jak i~niefunkcjonalnych). Jego zadaniem jest również dbałość o~ich prawidłową realizację. To głównie z~analitykami będzie współpracował \emph{Product Owner}.\\

Główne zadania:
\begin{itemize}
  \item analiza środowiska systemowego,
  \item określenie wymagań stawianych przed systemem,
  \item tworzenie specyfikacji wymagań,
  \item tworzenie dokumentacji projektowej,
  \item tworzenie planu testów.
\end{itemize}

\subsubsection{Projektanci -- wymagania}
Każdy z~projektantów będzie odpowiedzialny za stworzenie architektury projektowanego systemu oraz udokumentowanie jej. Wyniki prac tego zespołu będą kluczowe dla prac programistów implementujących właściwy system.\\

Główne zadania:
\begin{itemize}
  \item utworzenie projektu systemu,
  \item utworzenie dokumentacji technicznej projektu systemu,
  \item wybór technologii oraz metod realizacji systemu,
  \item dostosowywanie wymagań do postępów prac.
\end{itemize}

