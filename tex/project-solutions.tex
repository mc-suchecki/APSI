\section{Rozwiązania projektowe}

\subsection {Środowisko}
Obecnie coraz większa część systemów informatycznych realizowana jest w postaci aplikacji z~dostępem z~poziomu przeglądarki internetowej. Taki zabieg pozwala na~bardzo łatwy dostęp do~systemu z~poziomu praktycznie dowolnego urządzenia posiadającego dostęp do~internetu. Innymi kluczowym czynnikiem jest duża łatwość w~dystrybucji takiego rozwiązania i~w~związku z~tym klient również zdecydował się na~zastosowanie takiego podejścia.\\

Środowiskiem pracy dla użytkowników tworzonego systemu będzie przeglądarka internetowa. Dzięki temu klienci będą mogli korzystać z dostarczonego systemu zarówno przy użyciu komputera osobistego jak i~urządzenia mobilnego. Rozwiązanie takie jak dedykowane aplikacje mogą być przydatne na~niektórych rodzajach urządzeń, jednak tworzenie ich na~wszystkie możliwe rynki (stacjonarne, mobile itp.) stanowiłoby duże wyzwanie i~spowodowało znaczące przekroczenia zarówno budżetu jak i harmonogramu.\\

Zdecydowano się na~wsparcie następujących rodzajów przeglądarek:
\begin{enumerate}
  \item Google Chrome (od~wersji 23 wzwyż)
  \item Mozilla Firefox (od~wersji 21 wzwyż)
  \item Safari (od~wersji 6 wzwyż)
  \item Opera (od~wersji 15 wzwyż)
  \item Internet Explorer (od~wersji 10 wzwyż)
\end{enumerate}

Pozostałe przeglądarki także powinny poprawnie prezentować stronę internetową sklepu, jednak wsparcie dla nich nie jest wymaganiem, a~co~za~tym idzie, dla przeglądarek tych nie będą przeprowadzane testy.\\

Wygląd strony internetowej powinien być taki sam (z~różnicami maksymalnie 0.04\% zawartości) dla każdej przeglądarki internetowej. Ewentualne różnice wynikające na przykład z~różnicy w~formatach monitorów czy ich wielkości powinny być obsługiwane przez mechanizmy wewnętrzne.\\

Ewentualne aplikacje wspomagające korzystanie ze~sklepu (na~przykład zdobywające coraz większą popularność aplikacje na~urządzenia mobilne) nie znajdują się w~fazie analizy w~niniejszym projekcie, ewentualnie mogą zostać stworzone w~czasie rozbudowy i~utrzymywania systemu. Aby pozostawić możliwość tego rodzaju rozszerzeń należy zadbać o~odpowiedni protokół komunikacyjny uniezależniający działanie serwerów aplikacyjnych i~bazy danych od~klienta, który dostarcza dane i~polecenia.

\subsection{Architektura}
% TODO Kuba

\subsection{Sprzęt}
% TODO Kombajn

