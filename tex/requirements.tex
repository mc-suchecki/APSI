\section{Wymagania}
W tej sekcji znajduje się lista wymagań, jakie spełniać powinien budowany system. Podane są~one z~podziałem na~dwie kategorie. Pierwsza to wymagania funkcjonalne -- określające funkcjonalności systemu oraz sposoby ich użycia. Natomiast druga to~wymagania niefunkcjonalne, które opisują ilościowe i~jakościowe warunki działania systemu.

\subsection{Wymagania funkcjonalne}
\subsubsection{Użytkownicy}
Wymagania funkcjonalne dotyczące zarządzaniem użytkownikami w~systemie.
\begin{itemize}
  \item logowanie do aplikacji,
  \item wyświetlanie listy użytkowników,
  \item dodawanie nowego użytkownika,
  \item edycja danych użytkownika,
  \item usunięcie użytkownika.
\end{itemize}

\subsubsection{Konta}
Wymagania funkcjonalne dotyczące zarządzaniem kontami pieniężnymi w~systemie.
\begin{itemize}
  \item wyświetlanie listy kont,
  \item dodawanie nowego konta,
  \item edycja danych konta,
  \item usunięcie konta.
\end{itemize}

\subsubsection{Transakcje}
Wymagania funkcjonalne dotyczące zarządzaniem transakcjami w~systemie.
\begin{itemize}
  \item wyświetlanie listy transakcji,
  \item dodawanie nowej transakcji,
  \item edycja danych transakcji,
  \item usunięcie transakcji.
\end{itemize}

\subsubsection{Budżet}
Wymagania funkcjonalne dotyczące zarządzaniem budżetami w~systemie.
\begin{itemize}
  \item wyświetlenie budżetów ustawionych w~systemie,
  \item ustawienie budżetu dla określonego użytkownika,
  \item modyfikacja budżetu sumarycznego dla wszystkich użytkowników,
  \item modyfikacja budżetu określonego użytkownika,
  \item usunięcie budżetu określonego użytkownika.
\end{itemize}

\subsubsection{Raporty}
Wymagania funkcjonalne dotyczące generowania raportów w~systemie.
% TODO Kombajn
(generowanie raportów/wykresów)

\subsubsection{Opis przypadków użycia -- użytkownicy}
% TODO Kuba
\paragraph{Logowanie do~aplikacji}
\label{par:login}

\subsubsection{Opis przypadków użycia -- konta}
\paragraph{Wyświetlanie listy kont\newline}
\label{par:accountsView}
Korzysta z~\ref{par:login}~--~Logowanie do~aplikacji.\\
\indent Funkcja generalizująca dla~\ref{par:accountCreate},~\ref{par:accountEdit} oraz~\ref{par:accountDelete}.\\\\
\textit{Scenariusz główny}
\begin{enumerate}
  \item Użytkownik otwiera stronę ,,Konta''.
  \item Jeśli użytkownik nie jest zalogowany, aplikacja wymaga zalogowania, inicjując~\ref{par:login}~--~Logowanie do~aplikacji.
  \item Aplikacja wyświetla konta należące do zalogowanego użytkownika.
\end{enumerate}

\paragraph{Utworzenie nowego konta\newline}
\label{par:accountCreate}
Funkcja~specjalizująca~dla~\ref{par:accountsView}~--~Wyświetlanie listy kont.\\\\
\textit{Scenariusz główny:}
\begin{enumerate}
  \item[1-3.] Jak w~funkcji generalizującej~\ref{par:accountsView}~--~Wyświetlanie listy kont.
  \item[4.] Użytkownik klika przycisk ,,Dodaj nowe konto''.
  \item[5.] Aplikacja wyświetla okno zawierające pola z~danymi dla nowego konta.
  \item[6.] Użytkownik wprowadza nazwę konta, początkowe saldo, wybiera walutę, typ konta i~klika przycisk ,,Dodaj''.
  \item[7.] System weryfikuje wprowadzone dane (np. czy para użytkownik-nazwa jest unikatowa).
  \item[8.] Jeśli dane są~prawidłowe, system tworzy nowe konto i~powiadamia o~tym użytkownika.
\end{enumerate}
\textit{Scenariusz alternatywny -- niepoprawne dane:}
\begin{enumerate}
  \item[1-7.] Jak w~scenariuszu głównym.
  \item[8.] System wyświetla informację o wykrytym błędzie.
  \item[9.] Użytkownik poprawia dane i~wciska przycisk ,,Dodaj''.
  \item[10.] Powrót do kroku 7 ze~scenariusza głównego.
\end{enumerate}

\paragraph{Edycja danych konta\newline}
\label{par:accountEdit}
Funkcja~specjalizująca~dla~\ref{par:accountsView}~--~Wyświetlanie listy kont.\\\\
\textit{Scenariusz główny:}
\begin{enumerate}
  \item[1-3.] As in generalizing function~\ref{par:accountsView}~--~Viewing accounts list.
  \item[4.] Użytkownik zaznacza jedno z~kont i~klika przycisk ,,Edytuj'' obok niego.
  \item[5.] Aplikacja wyświetla okno zawierające dane zaznaczonego konta.
  \item[6.] Użytkownik edytuje nazwę konta, walutę lub typ konta i~klika przycisk ,,Zapisz''.
  \item[7.] System weryfikuje wprowadzone dane (np. czy para użytkownik-nazwa jest unikatowa).
  \item[8.] Jeśli dane są~prawidłowe, system edytuje konto i~powiadamia o~tym użytkownika.
\end{enumerate}
\textit{Scenariusz alternatywny -- niepoprawne dane:}
\begin{enumerate}
  \item[1-7.] Jak w~scenariuszu głównym.
  \item[8.] System wyświetla informację o wykrytym błędzie.
  \item[9.] Użytkownik poprawia dane i~wciska przycisk ,,Zapisz''.
  \item[10.] Powrót do kroku 7 ze~scenariusza głównego.
\end{enumerate}

\paragraph{Usunięcie konta\newline}
\label{par:accountDelete}
Funkcja~specjalizująca~dla~\ref{par:accountsView}~--~Wyświetlanie listy kont.\\\\
\textit{Scenariusz główny:}
\begin{enumerate}
  \item[1-3.] As in generalizing function~\ref{par:accountsView}~--~Viewing accounts list.
  \item[4.] Użytkownik zaznacza jedno z~kont i~klika przycisk ,,Usuń'' obok niego.
  \item[5.] Aplikacja wyświetla okno potwierdzenia.
  \item[6.] Użytkownik klika przycisk ,,Usuń'' w~wyświetlonym oknie.
  \item[7.] Aplikacja zamyka okno, usuwa konto i~odświeża listę kont.
\end{enumerate}
\textit{Scenariusz alternatywny -- przerwanie operacji przez użytkownika:}
\begin{enumerate}
  \item[1-5.] Jak w~scenariuszu głównym.
  \item[6.] Użytkownik wciska przycisk ,,Anuluj''.
  \item[7.] Aplikacja zamyka okno.
\end{enumerate}

\subsubsection{Opis przypadków użycia -- transakcje}
\paragraph{Wyświetlanie listy transakcji\newline}
\label{par:transactionsView}
Wykorzystuje~\ref{par:login}~--~Logowanie do~aplikacji.\\
\indent Funkcja generalizująca dla~\ref{par:transactionCreate},~\ref{par:transactionEdit} oraz~\ref{par:transactionDelete}.\\\\
\textit{Scenariusz główny:}
\begin{enumerate}
  \item Użytkownik otwiera stronę ,,Transakcje''.
  \item Jeśli użytkownik nie jest zalogowany, aplikacja wymaga zalogowania, inicjując~\ref{par:login}~--~Logowanie do~aplikacji.
  \item Aplikacja wyświetla listę transakcji powiązanych z kontem danego użytkownika.
\end{enumerate}
\textit{Scenariusz alternatywny -- filtrowanie transakcji:}
\begin{enumerate}
  \item[1-3.] Jak w~scenariuszu głównym.
  \item[4.] Użytkownik wprowadza parametry filtrowania jak nazwa użytkownika, kategoria albo data transakcji.
  \item[5.] Aplikacja odświeża listę transakcji, wyświetlając tylko te, które pasują do podanych filtrów.
\end{enumerate}

\paragraph{Utworzenie nowej transakcji\newline}
\label{par:transactionCreate}
Funkcja~specjalizująca~dla~\ref{par:transactionsView}~--~Wyświetlanie listy transakcji.\\\\
\textit{Scenariusz główny:}
\begin{enumerate}
  \item[1-3.] Jak w~scenariuszu generalizującym~\ref{par:transactionsView}~--~Wyświetlenie listy transakcji.
  \item[4.] Użytkownik wciska przycisk ,,Dodaj transakcję''.
  \item[5.] Aplikacja wyświetla okno z~polami do~wprowadzania danych dotyczących transakcji.
  \item[6.] Użytkownik wyświetla rodzaj transakcji (wpływ/wydatek/transfer), konto od/do, wprowadza wartość, datę, opis i~klika przycisk ,,Dodaj''.
  \item[7.] System weryfikuje wprowadzone dane (np. czy na danym koncie znajduje się wystarczająca ilość pieniędzy).
  \item[8.] Jeśli dane są poprawne, system tworzy nową transakcję, przelicza ilość pieniędzy pozostałych na~koncie i~informuje użytkownika.
\end{enumerate}
\textit{Scenariusz alternatywny -- niepoprawne dane:}
\begin{enumerate}
  \item[1-7.] Jak w~scenariuszu głównym.
  \item[8.] System wyświetla informację o wykrytym błędzie.
  \item[9.] Użytkownik poprawia dane i~wciska przycisk ,,Dodaj''.
  \item[10.] Powrót do kroku 7 ze~scenariusza głównego.
\end{enumerate}

\paragraph{Edycja transakcji\newline}
\label{par:transactionEdit}
Funkcja~specjalizująca~dla~\ref{par:transactionsView}~--~Wyświetlanie listy transakcji.\\\\
\textit{Scenariusz główny:}
\begin{enumerate}
  \item[1-3.] Jak w~scenariuszu generalizującym~\ref{par:transactionsView}~--~Wyświetlenie listy transakcji.
  \item[4.] Użytkownik wybiera jedną z~transakcji i~wciska przycisk ,,Edytuj''.
  \item[5.] Aplikacja wyświetla okno z~informacjami o~transakcji.
  \item[6.] Użytkownik modyfikuje dane transakcji (wpływ/wydatek/transfer), konto od/do, wprowadza wartość, datę, opis i~klika przycisk ,,Zapisz''.
  \item[7.] System weryfikuje wprowadzone dane (np. czy na danym koncie znajduje się wystarczająca ilość pieniędzy).
  \item[8.] Jeśli dane są~poprawne, system tworzy nową transakcję, przelicza ilość pieniędzy pozostałych na~koncie i~informuje użytkownika.
\end{enumerate}
\textit{Scenariusz alternatywny -- niepoprawne dane:}
\begin{enumerate}
  \item[1-7.] Jak w~scenariuszu głównym.
  \item[8.] System wyświetla informację o~wykrytym błędzie i~czeka na~poprawkę.
  \item[9.] Użytkownik poprawia dane i~wciska przycisk ,,Zapisz''.
  \item[10.] Powrót do~kroku 7 ze~scenariusza głównego.
\end{enumerate}

\paragraph{Usuwanie transakcji\newline}
\label{par:transactionDelete}
Funkcja~specjalizująca~dla~\ref{par:transactionsView}~--~Wyświetlanie listy transakcji.\\\\
\textit{Scenariusz główny:}
\begin{enumerate}
  \item[1-3.] Jak w~scenariuszu generalizującym~\ref{par:transactionsView}~--~Wyświetlenie listy transakcji.
  \item[4.] Użytkownik wybiera jedną z~transakcji i~wciska przycisk ,,Usuń''.
  \item[5.] Aplikacja wyświetla okienko wymagające potwierdzenia operacji.
  \item[6.] Użytkownik potwierdza operację wciskając przycisk ,,Usuń''.
  \item[7.] Aplikacja zamyka okienko, usuwa transakcję, przelicza ilość pieniędzy na koncie i odświeża listę transakcji.
\end{enumerate}
\textit{Scenariusz alternatywny -- przerwanie operacji przez użytkownika:}
\begin{enumerate}
  \item[1-5.] Jak w~scenariuszu głównym.
  \item[6.] Użytkownik wciska przycisk ,,Anuluj''.
  \item[7.] Aplikacja zamyka okno.
\end{enumerate}

\subsubsection{Opis przypadków użycia -- budżet}
\paragraph{Wyświetlenie budżetów ustawionych w systemie\newline}
\label{par:budgetsList}
Wykorzystuje~\ref{par:login}~--~Logowanie do~aplikacji.\\
\indent Funkcja generalizująca dla~\ref{par:userBudget},~\ref{par:usersBudget},~\ref{par:userBudgetEdit} oraz~\ref{par:userBudgetDelete}.\\\\
\textit{Scenariusz główny:}
\begin{enumerate}
  \item Użytkownik otwiera stronę ,,Budżety''.
  \item Jeśli użytkownik nie jest zalogowany, aplikacja wymaga zalogowania, inicjując~\ref{par:login}~--~Logowanie do~aplikacji.
  \item Aplikacja wyświetla budżet sumaryczny dla wszystkich użytkowników oraz listę użytkowników z~budżetami dla każdego z~nich.
\end{enumerate}
\textit{Scenariusz alternatywny -- filtrowanie wyników:}
\begin{enumerate}
  \item[1-3.] Jak w~scenariuszu głównym.
  \item[4.] Użytkownik wprowadza nazwę konta, którego budżet ma~zostać wyświetlony i~wciska ,,Enter''.
  \item[5.] Aplikacja odświeża listę budżetów wyświetlając tylko jeden dla podanego użytkownika.
\end{enumerate}

\paragraph{Ustawienie budżetu dla określonego użytkownika\newline}
\label{par:userBudget}
Funkcja~specjalizująca~dla~\ref{par:budgetsList}~--~Wyświetlanie budżetów.\\\\
\textit{Scenariusz główny:}
\begin{enumerate}
  \item[1-3.] Jak w~scenariuszu generalizującym~\ref{par:budgetsList}~--~Wyświetlenie budżetów występujących w~systemie.
  \item[4.] Użytkownik wciska przycisk ,,Dodaj''.
  \item[5.] Aplikacja wyświetla okno z możliwością wyboru konta i~wprowadzenia nowego budżetu.
  \item[6.] Użytkownik wybiera konto, wprowadza nowy budżet i~wciska przycisk "Dodaj".
  \item[7.] Aplikacja weryfikuje poprawność wprowadzonych danych (np. czy liczba nie jest ujemna).
  \item[8.] Jeśli wartość jest poprawna system zapisuje nową wartość budżetu, zamyka okno i~aktualizuje stronę z~budżetami.
\end{enumerate}
\textit{Scenariusz alternatywny -- przerwanie operacji przez użytkownika:}
\begin{enumerate}
  \item[1-5.] Jak w~scenariuszu głównym.
  \item[6.] Użytkownik wciska przycisk "Anuluj".
  \item[7.] Aplikacja zamyka okno.
\end{enumerate}
\textit{Scenariusz alternatywny -- niepoprawne dane:}
\begin{enumerate}
  \item[1-7.] Jak w~scenariuszu głównym.
  \item[8.] System wyświetla informację o~niepoprawnych danych.
  \item[9.] Użytkownik wprowadza nową wartość i~wciska przycisk ,,Dodaj''.
  \item[10.] Powrót do kroku 7 ze~scenariusza głównego.
\end{enumerate}

\paragraph{Modyfikacja budżetu sumarycznego dla wszystkich użytkowników\newline}
\label{par:usersBudget}
Funkcja~specjalizująca~dla~\ref{par:budgetsList}~--~Wyświetlanie budżetów.\\\\
\textit{Scenariusz główny:}
\begin{enumerate}
  \item[1-3.] Jak w~scenariuszu generalizującym~\ref{par:budgetsList}~--~Wyświetlenie budżetów występujących w~systemie.
  \item[4.] Użytkownik wciska przycisk "Edytuj" obok pola wyświetlającego budżet sumaryczny.
  \item[5.] Aplikacja wyświetla okno z~możliwością wprowadzenia nowego budżetu sumarycznego.
  \item[6.] Użytkownik wprowadza nowy budżet i~wciska przycisk ,,Zapisz''.
  \item[7.] Aplikacja weryfikuje poprawność wprowadzonych danych (np. czy liczba nie jest ujemna).
  \item[8.] Jeśli wartość jest poprawna, system zapisuje nową wartość budżetu, zamyka okno i~aktualizuje stronę z budżetami.
\end{enumerate}
\textit{Scenariusz alternatywny -- przerwanie operacji przez użytkownika:}
\begin{enumerate}
  \item[1-5.] Jak w~scenariuszu głównym.
  \item[6.] Użytkownik wciska przycisk "Anuluj".
  \item[7.] Aplikacja zamyka okno.
\end{enumerate}
\textit{Scenariusz alternatywny -- niepoprawne dane:}
\begin{enumerate}
  \item[1-7.] Jak w~scenariuszu głównym.
  \item[8.] System wyświetla informację o~niepoprawnych danych.
  \item[9.] Użytkownik wprowadza nową wartość i~wciska przycisk ,,Zapisz''.
  \item[10.] Powrót do kroku 7 ze~scenariusza głównego.
\end{enumerate}

\paragraph{Modyfikacja budżetu dla określonego użytkownika\newline}
\label{par:userBudgetEdit}
Funkcja~specjalizująca~dla~\ref{par:budgetsList}~--~Wyświetlanie budżetów.\\\\
\textit{Scenariusz główny:}
\begin{enumerate}
  \item[1-3.] Jak w~scenariuszu generalizującym~\ref{par:budgetsList}~--~Wyświetlenie budżetów występujących w~systemie.
  \item[4.] Użytkownik wciska przycisk ,,Edytuj'' obok pola wyświetlającego budżet dla określonego użytkownika.
  \item[5.] Aplikacja wyświetla okno z~możliwością wprowadzenia nowego budżetu.
  \item[6.] Użytkownik wprowadza nowy budżet i~wciska przycisk ,,Zapisz''.
  \item[7.] Aplikacja weryfikuje poprawność wprowadzonych danych (np. czy liczba nie jest ujemna).
  \item[8.] Jeśli wartość jest poprawna, system zapisuje nową wartość budżetu, zamyka okno i~aktualizuje stronę z~budżetami.
\end{enumerate}
\textit{Scenariusz alternatywny -- przerwanie operacji przez użytkownika:}
\begin{enumerate}
  \item[1-5.] Jak w~scenariuszu głównym.
  \item[6.] Użytkownik wciska przycisk ,,Anuluj''.
  \item[7.] Aplikacja zamyka okno.
\end{enumerate}
\textit{Scenariusz alternatywny -- niepoprawne dane:}
\begin{enumerate}
  \item[1-7.] Jak w~scenariuszu głównym.
  \item[8.] System wyświetla informację o~niepoprawnych danych.
  \item[9.] Użytkownik wprowadza nową wartość i~wciska przycisk ,,Zapisz''.
  \item[10.] Powrót do kroku 7 ze~scenariusza głównego.
\end{enumerate}

\paragraph{Usunięcie budżetu dla określonego użytkownika\newline}
\label{par:userBudgetDelete}
Funkcja~specjalizująca~dla~\ref{par:budgetsList}~--~Wyświetlanie budżetów.\\\\
\textit{Scenariusz główny:}
\begin{enumerate}
  \item[1-3.] Jak w~scenariuszu generalizującym~\ref{par:budgetsList}~--~Wyświetlenie budżetów występujących w~systemie.
  \item[4.] Użytkownik wciska przycisk ,,Usuń'' obok pola wyświetlającego budżet dla określonego użytkownika.
  \item[5.] Aplikacja wyświetla okno wymagające potwierdzenia operacji.
  \item[6.] Użytkownik potwierdza operację wciskając przycisk ,,Usuń''.
  \item[7.] System usuwa budżet dla określonego użytkownika, zamyka okno i~odświeża stronę z~budżetami.
\end{enumerate}
\textit{Scenariusz alternatywny -- przerwanie operacji przez użytkownika:}
\begin{enumerate}
  \item[1-5.] Jak w~scenariuszu głównym.
  \item[6.] Użytkownik klika przycisk ,,Anuluj''.
  \item[7.] Aplikacja zamyka okno.
\end{enumerate}

\subsubsection{Opis przypadków użycia -- raporty}
% TODO Kombajn

\newpage
\subsection{Wymagania niefunkcjonalne}
% TODO Kombajn
