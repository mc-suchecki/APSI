\section{Wymagania}
W tej sekcji znajduje się lista wymagań, jakie spełniać powinien budowany system. Podane są~one z~podziałem na~dwie kategorie. Pierwsza to wymagania funkcjonalne -- określające funkcjonalności systemu oraz sposoby ich użycia. Natomiast druga to~wymagania niefunkcjonalne, które opisują ilościowe i~jakościowe warunki działania systemu.

\subsection{Wymagania funkcjonalne}
\subsubsection{Użytkownicy}
Wymagania funkcjonalne dotyczące zarządzaniem użytkownikami w~systemie.
\begin{itemize}
  \item logowanie do aplikacji,
  \item wyświetlanie listy użytkowników,
  \item dodawanie nowego użytkownika,
  \item edycja danych użytkownika,
  \item usunięcie użytkownika.
\end{itemize}

\subsubsection{Konta}
Wymagania funkcjonalne dotyczące zarządzaniem kontami pieniężnymi w~systemie.
\begin{itemize}
  \item wyświetlanie listy kont,
  \item dodawanie nowego konta,
  \item edycja danych konta,
  \item usunięcie konta.
\end{itemize}

\subsubsection{Transakcje}
Wymagania funkcjonalne dotyczące zarządzaniem transakcjami w~systemie.
\begin{itemize}
  \item wyświetlanie listy transakcji,
  \item dodawanie nowej transakcji,
  \item edycja danych transakcji,
  \item usunięcie transakcji.
\end{itemize}

\subsubsection{Budżet}
Wymagania funkcjonalne dotyczące zarządzaniem budżetami w~systemie.
\begin{itemize}
  \item wyświetlenie budżetów ustawionych w~systemie,
  \item ustawienie budżetu dla określonego użytkownika,
  \item modyfikacja budżetu sumarycznego dla wszystkich użytkowników,
  \item modyfikacja budżetu określonego użytkownika,
  \item usunięcie budżetu określonego użytkownika.
\end{itemize}

\subsubsection{Raporty}
Wymagania funkcjonalne dotyczące generowania raportów w~systemie.
% Kombajn
\begin{itemize}
  \item konfiguracja raportu
  \item wyświetlanie raportu
  \item zapisywanie raportu do pliku
\end{itemize}

\subsubsection{Opis przypadków użycia -- użytkownicy}

\paragraph{Dodawanie nowego użytkownika\newline}
\label{par:register}

\textit{Scenariusz główny}

\begin{enumerate}
\item Użytkownik otwiera stronę ,,Logowanie/Rejestracja''.
\item Użytkownik wybiera opcję ,,Rejestracja'' i~wprowadza dane: imię, nazwisko, adres e-mail, oraz hasło.
\item Użytkownik naciska przycisk ,,Zarejestruj''.
\item Aplikacja sprawdza wprowadzone dane.
\item Jeżeli nie wystąpiły żadne błędy, system wysyła aktywacyjną wiadomość e-mail na adres wprowadzony przez
użytkownika i~informuje go, aby go odebrał.
\item Użytkownik otwiera wiadomość i~klika w~link aktywacyjny.
\item Aplikacja rejestruje nowego użytkownika i~przekierowuje go do strony logowania.
\end{enumerate}

\textit{Scenariusz alternatywny -- niepoprawne dane}

\begin{enumerate}
\item[1-4.] Jak w~scenariuszu głównym.
\item[5.] System wyświetla informacje o~wykrytym błędzie.
\item[6.] Użytkownik poprawia wprowadzone dane i~naciska przycisk ,,Zarejestruj''.
\item[7.] Powrót do kroku 4.~ze scenariusza głównego.
\end{enumerate}

\paragraph{Logowanie do~aplikacji\newline}
\label{par:login}

\textit{Scenariusz główny}

\begin{enumerate}
\item Użytkownik otwiera stronę ,,Logowanie/Rejestracja''.
\item Użytkownik wybiera opcję ,,Logowanie'' i~wprowadza dane:~adres e-mail oraz hasło.
\item Użytkownik naciska przycisk ,,Zaloguj''.
\item Aplikacja sprawdza wprowadzone dane.
\item Jeżeli para (e-mail,~hasło) znajduje się w~systemie, aplikacja przekierowuje użytkownika na docelową stronę.
\end{enumerate}

\textit{Scenariusz alternatywny -- niepoprawne dane}

\begin{enumerate}
\item[1-4.] Jak w~scenariuszu głównym.
\item[5.] System wyświetla informacje o~wykrytym błędzie.
\item[6.] Użytkownik poprawia wprowadzone dane i~naciska przycisk ,,Login''.
\item[7.] Powrót do kroku 4.~ze scenariusza głównego.
\end{enumerate}

\paragraph{Wyświetlanie listy użytkowników\newline}
\label{par:userList}
Korzysta z~\ref{par:login}~--~Logowanie do~aplikacji.\\

\textit{Scenariusz główny}

\begin{enumerate}
\item Użytkownik otwiera stronę ,,Użytkownicy''.
\item Jeśli użytkownik nie jest zalogowany, aplikacja wymaga zalogowania, inicjując~\ref{par:login}~--~Logowanie do~aplikacji.
\item Aplikacja wyświetla użytkowników zarejestrowanych w~systemie.
\end{enumerate}

\paragraph{Edycja danych użytkownika\newline}
\label{par:editUser}
Korzysta z~\ref{par:login}~--~Logowanie do~aplikacji.\\

\textit{Scenariusz główny}

\begin{enumerate}
\item Użytkownik loguje się do aplikacji, tak jak w~scenariuszu~\ref{par:login}~--~Logowanie do aplikacji.
\item Użytkownik otwiera stronę ,,Profil''.
\item Użytkownik naciska przycisk ,,Edytuj profil''.
\item Aplikacja wyświetla okno zawierające dane użytkownika.
\item Użytkownik edytuje imię, nazwisko, adres e-mail lub hasło i~naciska przycisk ,,Zapisz''.
\item System prosi użytkownika o~potwierdzenie zmiany poprzez wpisanie starego hasła.
\item System weryfikuje wprowadzone dane.
\item Jeżeli dane są prawidłowe, system edytuje dane użytkownika.
\end{enumerate}

\textit{Scenariusz alternatywny -- niepoprawne dane}

\begin{enumerate}
\item[1-7.] Jak w~scenariuszu głównym.
\item[8.] System wyświetla informacje o~wykrytym błędzie.
\item[9.] Użytkownik poprawia dane i~naciska przycisk ,,Zapisz''.
\item[10.] Powrót do kroku 7.~ze scenariusza głównego.
\end{enumerate}

\paragraph{Usunięcie użytkownika\newline}
\label{par:deleteUser}
Korzysta z~\ref{par:login}~--~Logowanie do~aplikacji.\\

\textit{Scenariusz główny}

\begin{enumerate}
\item Użytkownik loguje się do aplikacji, tak jak w~scenariuszu~\ref{par:login}~--~Logowanie do aplikacji.
\item Użytkownik otwiera stronę ,,Profil''.
\item Użytkownik naciska przycisk ,,Usuń użytkownika''.
\item System prosi użytkownika o~potwierdzenie decyzji poprzez wpisanie hasła i~naciśnięcie przycisku ,,Usuń''.
\item System weryfikuje poprawność hasła.
\item Jeżeli hasło jest zgodne, aplikacja usuwa informacje o~użytkowniku i~następuje wylogowanie.
\end{enumerate}

\textit{Scenariusz alternatywny -- użytkownik nie potwierdza decyzji}

\begin{enumerate}
\item[1-5.] Jak w~scenariuszu głównym.
\item[6.] Użytkownik nacisnął przycisk ,,Anuluj''.
\item[7.] System powraca do strony ,,Profil''.
\end{enumerate}

\textit{Scenariusz alternatywny -- niepoprawne dane}

\begin{enumerate}
\item[1-5.] Jak w~scenariuszu głównym.
\item[6.] System wyświetla informację o~wykrytym błędzie.
\item[7.] Użytkownik poprawia hasło i~naciska przycisk ,,Usuń''.
\item[7.] Powrót do kroku 5.~ze scenariusza głównego.
\end{enumerate}

\subsubsection{Opis przypadków użycia -- konta}
\paragraph{Wyświetlanie listy kont\newline}
\label{par:accountsView}
Korzysta z~\ref{par:login}~--~Logowanie do~aplikacji.\\
\indent Funkcja generalizująca dla~\ref{par:accountCreate},~\ref{par:accountEdit} oraz~\ref{par:accountDelete}.\\\\
\textit{Scenariusz główny}
\begin{enumerate}
  \item Użytkownik otwiera stronę ,,Konta''.
  \item Jeśli użytkownik nie jest zalogowany, aplikacja wymaga zalogowania, inicjując~\ref{par:login}~--~Logowanie do~aplikacji.
  \item Aplikacja wyświetla konta należące do zalogowanego użytkownika.
\end{enumerate}

\paragraph{Utworzenie nowego konta\newline}
\label{par:accountCreate}
Funkcja~specjalizująca~dla~\ref{par:accountsView}~--~Wyświetlanie listy kont.\\\\
\textit{Scenariusz główny:}
\begin{enumerate}
  \item[1-3.] Jak w~funkcji generalizującej~\ref{par:accountsView}~--~Wyświetlanie listy kont.
  \item[4.] Użytkownik klika przycisk ,,Dodaj nowe konto''.
  \item[5.] Aplikacja wyświetla okno zawierające pola z~danymi dla nowego konta.
  \item[6.] Użytkownik wprowadza nazwę konta, początkowe saldo, wybiera walutę, typ konta i~klika przycisk ,,Dodaj''.
  \item[7.] System weryfikuje wprowadzone dane (np. czy para użytkownik-nazwa jest unikatowa).
  \item[8.] Jeśli dane są~prawidłowe, system tworzy nowe konto i~powiadamia o~tym użytkownika.
\end{enumerate}
\textit{Scenariusz alternatywny -- niepoprawne dane:}
\begin{enumerate}
  \item[1-7.] Jak w~scenariuszu głównym.
  \item[8.] System wyświetla informację o wykrytym błędzie.
  \item[9.] Użytkownik poprawia dane i~wciska przycisk ,,Dodaj''.
  \item[10.] Powrót do kroku 7 ze~scenariusza głównego.
\end{enumerate}

\paragraph{Edycja danych konta\newline}
\label{par:accountEdit}
Funkcja~specjalizująca~dla~\ref{par:accountsView}~--~Wyświetlanie listy kont.\\\\
\textit{Scenariusz główny:}
\begin{enumerate}
  \item[1-3.] Jak w~funkcji generalizującej~\ref{par:accountsView}~--~Wyświetlanie listy kont.
  \item[4.] Użytkownik zaznacza jedno z~kont i~klika przycisk ,,Edytuj'' obok niego.
  \item[5.] Aplikacja wyświetla okno zawierające dane zaznaczonego konta.
  \item[6.] Użytkownik edytuje nazwę konta, walutę lub typ konta i~klika przycisk ,,Zapisz''.
  \item[7.] System weryfikuje wprowadzone dane (np. czy para użytkownik-nazwa jest unikatowa).
  \item[8.] Jeśli dane są~prawidłowe, system edytuje konto i~powiadamia o~tym użytkownika.
\end{enumerate}
\textit{Scenariusz alternatywny -- niepoprawne dane:}
\begin{enumerate}
  \item[1-7.] Jak w~scenariuszu głównym.
  \item[8.] System wyświetla informację o wykrytym błędzie.
  \item[9.] Użytkownik poprawia dane i~wciska przycisk ,,Zapisz''.
  \item[10.] Powrót do kroku 7 ze~scenariusza głównego.
\end{enumerate}

\paragraph{Usunięcie konta\newline}
\label{par:accountDelete}
Funkcja~specjalizująca~dla~\ref{par:accountsView}~--~Wyświetlanie listy kont.\\\\
\textit{Scenariusz główny:}
\begin{enumerate}
  \item[1-3.] Jak w~funkcji generalizującej~\ref{par:accountsView}~--~Wyświetlanie listy kont.
  \item[4.] Użytkownik zaznacza jedno z~kont i~klika przycisk ,,Usuń'' obok niego.
  \item[5.] Aplikacja wyświetla okno potwierdzenia.
  \item[6.] Użytkownik klika przycisk ,,Usuń'' w~wyświetlonym oknie.
  \item[7.] Aplikacja zamyka okno, usuwa konto i~odświeża listę kont.
\end{enumerate}
\textit{Scenariusz alternatywny -- przerwanie operacji przez użytkownika:}
\begin{enumerate}
  \item[1-5.] Jak w~scenariuszu głównym.
  \item[6.] Użytkownik wciska przycisk ,,Anuluj''.
  \item[7.] Aplikacja zamyka okno.
\end{enumerate}

\subsubsection{Opis przypadków użycia -- transakcje}
\label{par:transactionsView}
\paragraph{Wyświetlanie listy transakcji\newline}
Opis słowny - użytkownik będzie chciał wyświetlić listę transakcji zarejestrowanych w systemie.
Oprócz zwykłej listy zebranych danych, potrzebna jest możliwość filtrowania, co ułatwi użytkownikowi
przeglądanie listy.
\begin{longtable}{|p{5cm}|p{7cm}|}
 	\hline
	\textbf{Aktor} & Użytkownik \\
	\hline
	\textbf{Warunki początkowe} & Brak
	\\
	\hline
	\textbf{Opis przebiegu interakcji} & Wybór strony z transakcjami,
	wprowadzenie parametrów filtrowania
	\\
	\hline
	\textbf{Sytuacje wyjątkowe} & Żądanie filtrowania danych
	\\
	\hline
	\textbf{Warunki końcowe} & Wyświetlona lista żądanych transakcji
	\\
	\hline
 \end{longtable}
Wykorzystuje~\ref{par:login}~--~Logowanie do~aplikacji.\\
\indent Funkcja generalizująca dla~\ref{par:transactionCreate},~\ref{par:transactionEdit} oraz~\ref{par:transactionDelete}.\\\\
\textit{Scenariusz główny:}
\begin{enumerate}
  \item Użytkownik otwiera stronę ,,Transakcje''.
  \item Jeśli użytkownik nie jest zalogowany, aplikacja wymaga zalogowania, inicjując~\ref{par:login}~--~Logowanie do~aplikacji.
  \item Aplikacja wyświetla listę transakcji powiązanych z kontem danego użytkownika.
\end{enumerate}
\textit{Scenariusz alternatywny -- filtrowanie transakcji:}
\begin{enumerate}
  \item[1-3.] Jak w~scenariuszu głównym.
  \item[4.] Użytkownik wprowadza parametry filtrowania jak nazwa użytkownika, kategoria albo data transakcji.
  \item[5.] Aplikacja odświeża listę transakcji, wyświetlając tylko te, które pasują do podanych filtrów.
\end{enumerate}

\begin{figure}[H]
    \includegraphics[width=\textwidth,
    height=0.5\textheight]{images/wyswietl_transakcje.png}
    \caption{Diagram sekwencji dla przypadku użycia~\ref{par:transactionsView}~--~Wyświetlenie listy transakcji.
    - scenariusz główny}
\end{figure}

\paragraph{Utworzenie nowej transakcji\newline}
\label{par:transactionCreate}
\paragraph{Wyświetlanie listy transakcji\newline}
Opis słowny - jedna z podstawowych funkcjonalności systemu to dodawanie
transkacji, których koszty będą sumowane i nadzorowane. Akcje te mają miejsce
gdy użytkownik chce dodać wydatek do systemu.
\begin{longtable}{|p{5cm}|p{7cm}|}
 	\hline
	\textbf{Aktor} & Użytkownik \\
	\hline
	\textbf{Warunki początkowe} & Brak
	\\
	\hline
	\textbf{Opis przebiegu interakcji} & Wybór strony z transakcjami i wciśnięcie dodaj,
	uzupełnienie danych i zapisanie
	\\
	\hline
	\textbf{Sytuacje wyjątkowe} & Wprowadzenie niepoprawnych danych
	\\
	\hline
	\textbf{Warunki końcowe} & Dodanie nowej transkacji
	\\
	\hline
 \end{longtable}
Funkcja~specjalizująca~dla~\ref{par:transactionsView}~--~Wyświetlanie listy transakcji.\\\\
\textit{Scenariusz główny:}
\begin{enumerate}
  \item[1-3.] Jak w~scenariuszu generalizującym~\ref{par:transactionsView}~--~Wyświetlenie listy transakcji.
  \item[4.] Użytkownik wciska przycisk ,,Dodaj transakcję''.
  \item[5.] Aplikacja wyświetla okno z~polami do~wprowadzania danych dotyczących transakcji.
  \item[6.] Użytkownik wprowadza rodzaj transakcji (wpływ/wydatek/transfer), konto od/do, wprowadza wartość, datę, opis i~klika przycisk ,,Dodaj''.
  \item[7.] System weryfikuje wprowadzone dane (np. czy na danym koncie znajduje się wystarczająca ilość pieniędzy).
  \item[8.] Jeśli dane są poprawne, system tworzy nową transakcję, przelicza ilość pieniędzy pozostałych na~koncie i~informuje użytkownika.
\end{enumerate}
\textit{Scenariusz alternatywny -- niepoprawne dane:}
\begin{enumerate}
  \item[1-7.] Jak w~scenariuszu głównym.
  \item[8.] System wyświetla informację o wykrytym błędzie.
  \item[9.] Użytkownik poprawia dane i~wciska przycisk ,,Dodaj''.
  \item[10.] Powrót do kroku 7 ze~scenariusza głównego.
\end{enumerate}

\begin{figure}[H]
    \includegraphics[width=\textwidth,
    height=0.5\textheight]{images/dodanie_transakcji.png}
    \caption{Diagram sekwencji dla przypadku użycia~\ref{par:transactionCreate}~--~Utworzenie nowej transakcji.
    - scenariusz główny}
\end{figure}

\paragraph{Edycja transakcji\newline}
\label{par:transactionEdit}
Opis słowny - modyfikacja danych dotyczących wybranej transkacji może być
przydatna w sytuacji, gdy użytkownik wprowdzi błędne dane w kontekście logicznym,
ale które będą poprawnie w sensie formalnym.
\begin{longtable}{|p{5cm}|p{7cm}|}
 	\hline
	\textbf{Aktor} & Użytkownik \\
	\hline
	\textbf{Warunki początkowe} & Istnieją transkacje w systemie
	\\
	\hline
	\textbf{Opis przebiegu interakcji} & Wybór strony z transakcjami, wybór żądanej transkacji,
	modyfiakcja danych i zapisanie.
	\\
	\hline
	\textbf{Sytuacje wyjątkowe} & Wprowadzenie niepoprawnych danych
	\\
	\hline
	\textbf{Warunki końcowe} & Dodanie nowej transkacji
	\\
	\hline
 \end{longtable}
Funkcja~specjalizująca~dla~\ref{par:transactionsView}~--~Wyświetlanie listy transakcji.\\\\
\textit{Scenariusz główny:}
\begin{enumerate}
  \item[1-3.] Jak w~scenariuszu generalizującym~\ref{par:transactionsView}~--~Wyświetlenie listy transakcji.
  \item[4.] Użytkownik wybiera jedną z~transakcji i~wciska przycisk ,,Edytuj''.
  \item[5.] Aplikacja wyświetla okno z~informacjami o~transakcji.
  \item[6.] Użytkownik modyfikuje dane transakcji (wpływ/wydatek/transfer), konto od/do, wprowadza wartość, datę, opis i~klika przycisk ,,Zapisz''.
  \item[7.] System weryfikuje wprowadzone dane (np. czy na danym koncie znajduje się wystarczająca ilość pieniędzy).
  \item[8.] Jeśli dane są~poprawne, system zapisuje transakcję, przelicza ilość pieniędzy pozostałych na~koncie i~informuje użytkownika.
\end{enumerate}
\textit{Scenariusz alternatywny -- niepoprawne dane:}
\begin{enumerate}
  \item[1-7.] Jak w~scenariuszu głównym.
  \item[8.] System wyświetla informację o~wykrytym błędzie i~czeka na~poprawkę.
  \item[9.] Użytkownik poprawia dane i~wciska przycisk ,,Zapisz''.
  \item[10.] Powrót do~kroku 7 ze~scenariusza głównego.
\end{enumerate}

\begin{figure}[H]
    \includegraphics[width=\textwidth,
    height=0.5\textheight]{images/edycja_transakcji.png}
    \caption{Diagram sekwencji dla przypadku użycia~\ref{par:transactionEdit}~--~Edycja transakcji.
    - scenariusz główny}
\end{figure}

\paragraph{Usuwanie transakcji\newline}
\label{par:transactionDelete}
Opis słowny - akcje opisane poniżej mają miejsce w przypadku gdy np. użytkownik zwrócił
towar dotyczącego którą transkację już wprowadził.
\begin{longtable}{|p{5cm}|p{7cm}|}
 	\hline
	\textbf{Aktor} & Użytkownik \\
	\hline
	\textbf{Warunki początkowe} & Istnieją transkacje w systemie
	\\
	\hline
	\textbf{Opis przebiegu interakcji} & Wybór strony z transakcjami, wybór żądanej transkacji,
	usunięcie jej.
	\\
	\hline
	\textbf{Sytuacje wyjątkowe} & Brak
	\\
	\hline
	\textbf{Warunki końcowe} & Usunięcie transakcji
	\\
	\hline
 \end{longtable}
Funkcja~specjalizująca~dla~\ref{par:transactionsView}~--~Wyświetlanie listy transakcji.\\\\
\textit{Scenariusz główny:}
\begin{enumerate}
  \item[1-3.] Jak w~scenariuszu generalizującym~\ref{par:transactionsView}~--~Wyświetlenie listy transakcji.
  \item[4.] Użytkownik wybiera jedną z~transakcji i~wciska przycisk ,,Usuń''.
  \item[5.] Aplikacja wyświetla okienko wymagające potwierdzenia operacji.
  \item[6.] Użytkownik potwierdza operację wciskając przycisk ,,Usuń''.
  \item[7.] Aplikacja zamyka okienko, usuwa transakcję, przelicza ilość pieniędzy na koncie i odświeża listę transakcji.
\end{enumerate}
\textit{Scenariusz alternatywny -- przerwanie operacji przez użytkownika:}
\begin{enumerate}
  \item[1-5.] Jak w~scenariuszu głównym.
  \item[6.] Użytkownik wciska przycisk ,,Anuluj''.
  \item[7.] Aplikacja zamyka okno.
\end{enumerate}

\begin{figure}[H]
    \includegraphics[width=\textwidth,
    height=0.5\textheight]{images/usun_transakcje.png}
    \caption{Diagram sekwencji dla przypadku użycia~\ref{par:transactionDelete}~--~Usuwanie transakcji.
    - scenariusz główny}
\end{figure}

\subsubsection{Opis przypadków użycia -- budżet}
\paragraph{Wyświetlenie budżetów ustawionych w systemie\newline}
\label{par:budgetsList}
Opis słowny - głównym zadaniem systemu jest nadzorowanie domowego budżetu. Użytkownik
będzie chciał przejrzeć aktualny stan budżetów sumaryczny oraz poszczególny dla każdego z 
zarejestrowanych użytkowników.
\begin{longtable}{|p{5cm}|p{7cm}|}
 	\hline
	\textbf{Aktor} & Użytkownik \\
	\hline
	\textbf{Warunki początkowe} & Brak
	\\
	\hline
	\textbf{Opis przebiegu interakcji} & Wybór strony z budżetami
	\\
	\hline
	\textbf{Sytuacje wyjątkowe} & Żądanie filtrowania danych
	\\
	\hline
	\textbf{Warunki końcowe} & Wyświetlenie żądanych budżetów
	\\
	\hline
 \end{longtable}
Wykorzystuje~\ref{par:login}~--~Logowanie do~aplikacji.\\
\indent Funkcja generalizująca dla~\ref{par:userBudget},~\ref{par:usersBudget},~\ref{par:userBudgetEdit} oraz~\ref{par:userBudgetDelete}.\\\\
\textit{Scenariusz główny:}
\begin{enumerate}
  \item Użytkownik otwiera stronę ,,Budżety''.
  \item Jeśli użytkownik nie jest zalogowany, aplikacja wymaga zalogowania, inicjując~\ref{par:login}~--~Logowanie do~aplikacji.
  \item Aplikacja wyświetla budżet sumaryczny dla wszystkich użytkowników oraz listę użytkowników z~budżetami dla każdego z~nich.
\end{enumerate}
\textit{Scenariusz alternatywny -- filtrowanie wyników:}
\begin{enumerate}
  \item[1-3.] Jak w~scenariuszu głównym.
  \item[4.] Użytkownik wprowadza nazwę konta, którego budżet ma~zostać wyświetlony i~wciska ,,Enter''.
  \item[5.] Aplikacja odświeża listę budżetów wyświetlając tylko jeden dla podanego użytkownika.
\end{enumerate}

\paragraph{Ustawienie budżetu dla określonego użytkownika\newline}
\label{par:userBudget}
Opis słowny - celem lepszej kontroli wydatków, użytkownik może chcieć ustawić limit
budżetu dla określonego użytkownika, aby ten nie mógł przekroczyć określonej kwoty.
\begin{longtable}{|p{5cm}|p{7cm}|}
 	\hline
	\textbf{Aktor} & Użytkownik \\
	\hline
	\textbf{Warunki początkowe} & Brak
	\\
	\hline
	\textbf{Opis przebiegu interakcji} & Wybór strony z budżetami, wprowadzenie wymaganych danych,
	zapisanie nowego budżetu
	\\
	\hline
	\textbf{Sytuacje wyjątkowe} & Błędne wprowadzenie danych, przerwanie operacji
	\\
	\hline
	\textbf{Warunki końcowe} & Dodanie nowego limitu dla użytkownika
	\\
	\hline
 \end{longtable}
Funkcja~specjalizująca~dla~\ref{par:budgetsList}~--~Wyświetlanie budżetów.\\\\
\textit{Scenariusz główny:}
\begin{enumerate}
  \item[1-3.] Jak w~scenariuszu generalizującym~\ref{par:budgetsList}~--~Wyświetlenie budżetów występujących w~systemie.
  \item[4.] Użytkownik wciska przycisk ,,Dodaj''.
  \item[5.] Aplikacja wyświetla okno z możliwością wyboru konta i~wprowadzenia nowego budżetu.
  \item[6.] Użytkownik wybiera konto, wprowadza nowy budżet i~wciska przycisk "Dodaj".
  \item[7.] Aplikacja weryfikuje poprawność wprowadzonych danych (np. czy liczba nie jest ujemna).
  \item[8.] Jeśli wartość jest poprawna system zapisuje nową wartość budżetu, zamyka okno i~aktualizuje stronę z~budżetami.
\end{enumerate}
\textit{Scenariusz alternatywny -- przerwanie operacji przez użytkownika:}
\begin{enumerate}
  \item[1-5.] Jak w~scenariuszu głównym.
  \item[6.] Użytkownik wciska przycisk "Anuluj".
  \item[7.] Aplikacja zamyka okno.
\end{enumerate}
\textit{Scenariusz alternatywny -- niepoprawne dane:}
\begin{enumerate}
  \item[1-7.] Jak w~scenariuszu głównym.
  \item[8.] System wyświetla informację o~niepoprawnych danych.
  \item[9.] Użytkownik wprowadza nową wartość i~wciska przycisk ,,Dodaj''.
  \item[10.] Powrót do kroku 7 ze~scenariusza głównego.
\end{enumerate}

\begin{figure}[H]
    \includegraphics[width=\textwidth,
    height=0.5\textheight]{images/dodanie_budzetu_dla_usera.png}
    \caption{Diagram sekwencji dla przypadku użycia~\ref{par:userBudget}~--~Ustawienie budżetu dla określonego użytkownika.
    - scenariusz główny}
\end{figure}

\paragraph{Modyfikacja budżetu sumarycznego dla wszystkich użytkowników\newline}
\label{par:usersBudget}
Opis słowny - oprócz ustawiania limitu miesięcznego dla danego użytkownika, kluczowym
elementem tego systemu jest modyfikacja  limitu sumarycznego dla wszystkich użytkowników
danej aplikacji. Wartość ta musi być zawsze ustawiona, lecz użytkownik może ją dowolnie modyfikować.
\begin{longtable}{|p{5cm}|p{7cm}|}
 	\hline
	\textbf{Aktor} & Użytkownik \\
	\hline
	\textbf{Warunki początkowe} & Brak
	\\
	\hline
	\textbf{Opis przebiegu interakcji} & Wybór strony z budżetami, wprowadzenie wymaganych danych,
	zapisanie budżetu
	\\
	\hline
	\textbf{Sytuacje wyjątkowe} & Błędne wprowadzenie danych, przerwanie operacji
	\\
	\hline
	\textbf{Warunki końcowe} & Zmodyfikowanie budżetu sumarycznego
	\\
	\hline
 \end{longtable}
Funkcja~specjalizująca~dla~\ref{par:budgetsList}~--~Wyświetlanie budżetów.\\\\
\textit{Scenariusz główny:}
\begin{enumerate}
  \item[1-3.] Jak w~scenariuszu generalizującym~\ref{par:budgetsList}~--~Wyświetlenie budżetów występujących w~systemie.
  \item[4.] Użytkownik wciska przycisk "Edytuj" obok pola wyświetlającego budżet sumaryczny.
  \item[5.] Aplikacja wyświetla okno z~możliwością wprowadzenia nowego budżetu sumarycznego.
  \item[6.] Użytkownik wprowadza nowy budżet i~wciska przycisk ,,Zapisz''.
  \item[7.] Aplikacja weryfikuje poprawność wprowadzonych danych (np. czy liczba nie jest ujemna).
  \item[8.] Jeśli wartość jest poprawna, system zapisuje nową wartość budżetu, zamyka okno i~aktualizuje stronę z budżetami.
\end{enumerate}
\textit{Scenariusz alternatywny -- przerwanie operacji przez użytkownika:}
\begin{enumerate}
  \item[1-5.] Jak w~scenariuszu głównym.
  \item[6.] Użytkownik wciska przycisk "Anuluj".
  \item[7.] Aplikacja zamyka okno.
\end{enumerate}
\textit{Scenariusz alternatywny -- niepoprawne dane:}
\begin{enumerate}
  \item[1-7.] Jak w~scenariuszu głównym.
  \item[8.] System wyświetla informację o~niepoprawnych danych.
  \item[9.] Użytkownik wprowadza nową wartość i~wciska przycisk ,,Zapisz''.
  \item[10.] Powrót do kroku 7 ze~scenariusza głównego.
\end{enumerate}

\begin{figure}[H]
    \includegraphics[width=\textwidth,
    height=0.5\textheight]{images/modyfikacja_budzetu_sumarycznego.png}
    \caption{Diagram sekwencji dla przypadku użycia~\ref{par:usersBudget}~--~Ustawienie budżetu sumarycznego.
    - scenariusz główny}
\end{figure}

\paragraph{Modyfikacja budżetu dla określonego użytkownika\newline}
\label{par:userBudgetEdit}
Opis słowny - system musi pozwolić również modyfikować wcześniej ustawiony budżet
dla poszczególnych użytkowników, gdyż limity te mogą się zmieniać np. jak dziecko
dostanie większe kieszonkowe.
\begin{longtable}{|p{5cm}|p{7cm}|}
 	\hline
	\textbf{Aktor} & Użytkownik \\
	\hline
	\textbf{Warunki początkowe} & Brak
	\\
	\hline
	\textbf{Opis przebiegu interakcji} & Wybór strony z budżetami, wprowadzenie wymaganych danych,
	zapisanie budżetu
	\\
	\hline
	\textbf{Sytuacje wyjątkowe} & Błędne wprowadzenie danych, przerwanie operacji
	\\
	\hline
	\textbf{Warunki końcowe} & Zmodyfikowanie budżetu użytkownika
	\\
	\hline
 \end{longtable}
Funkcja~specjalizująca~dla~\ref{par:budgetsList}~--~Wyświetlanie budżetów.\\\\
\textit{Scenariusz główny:}
\begin{enumerate}
  \item[1-3.] Jak w~scenariuszu generalizującym~\ref{par:budgetsList}~--~Wyświetlenie budżetów występujących w~systemie.
  \item[4.] Użytkownik wciska przycisk ,,Edytuj'' obok pola wyświetlającego budżet dla określonego użytkownika.
  \item[5.] Aplikacja wyświetla okno z~możliwością wprowadzenia nowego budżetu.
  \item[6.] Użytkownik wprowadza nowy budżet i~wciska przycisk ,,Zapisz''.
  \item[7.] Aplikacja weryfikuje poprawność wprowadzonych danych (np. czy liczba nie jest ujemna).
  \item[8.] Jeśli wartość jest poprawna, system zapisuje nową wartość budżetu, zamyka okno i~aktualizuje stronę z~budżetami.
\end{enumerate}
\textit{Scenariusz alternatywny -- przerwanie operacji przez użytkownika:}
\begin{enumerate}
  \item[1-5.] Jak w~scenariuszu głównym.
  \item[6.] Użytkownik wciska przycisk ,,Anuluj''.
  \item[7.] Aplikacja zamyka okno.
\end{enumerate}
\textit{Scenariusz alternatywny -- niepoprawne dane:}
\begin{enumerate}
  \item[1-7.] Jak w~scenariuszu głównym.
  \item[8.] System wyświetla informację o~niepoprawnych danych.
  \item[9.] Użytkownik wprowadza nową wartość i~wciska przycisk ,,Zapisz''.
  \item[10.] Powrót do kroku 7 ze~scenariusza głównego.
\end{enumerate}

\begin{figure}[H]
    \includegraphics[width=\textwidth,
    height=0.5\textheight]{images/modyfikacja_budzetu_uzytkownika.png}
    \caption{Diagram sekwencji dla przypadku użycia~\ref{par:userBudgetEdit}~--~Modyfikacja budżetu dla określonego użytkownika.
    - scenariusz główny}
\end{figure}

\paragraph{Usunięcie budżetu dla określonego użytkownika\newline}
\label{par:userBudgetDelete}
Opis słowny - akcje te będą wykonane gdy np. użytkownik nie będzie dłużej korzystał
z podanej aplikacji lub po prostu limit miesięczny go przestał dotyczyć. Również może się
zmienić polityka i zamiast ustawiania budżetów na poszczególne osoby, zacznie obowiązywać
tylko budżet sumaryczny.
\begin{longtable}{|p{5cm}|p{7cm}|}
 	\hline
	\textbf{Aktor} & Użytkownik \\
	\hline
	\textbf{Warunki początkowe} & Istnieje limit na użytkownika.
	\\
	\hline
	\textbf{Opis przebiegu interakcji} & Wybór strony z budżetami, usunięcie wybranego budżetu
	\\
	\hline
	\textbf{Sytuacje wyjątkowe} & Przerwanie operacji
	\\
	\hline
	\textbf{Warunki końcowe} & Usunięcie budżetu użytkownika
	\\
	\hline
 \end{longtable}
Funkcja~specjalizująca~dla~\ref{par:budgetsList}~--~Wyświetlanie budżetów.\\\\
\textit{Scenariusz główny:}
\begin{enumerate}
  \item[1-3.] Jak w~scenariuszu generalizującym~\ref{par:budgetsList}~--~Wyświetlenie budżetów występujących w~systemie.
  \item[4.] Użytkownik wciska przycisk ,,Usuń'' obok pola wyświetlającego budżet dla określonego użytkownika.
  \item[5.] Aplikacja wyświetla okno wymagające potwierdzenia operacji.
  \item[6.] Użytkownik potwierdza operację wciskając przycisk ,,Usuń''.
  \item[7.] System usuwa budżet dla określonego użytkownika, zamyka okno i~odświeża stronę z~budżetami.
\end{enumerate}
\textit{Scenariusz alternatywny -- przerwanie operacji przez użytkownika:}
\begin{enumerate}
  \item[1-5.] Jak w~scenariuszu głównym.
  \item[6.] Użytkownik klika przycisk ,,Anuluj''.
  \item[7.] Aplikacja zamyka okno.
\end{enumerate}

\begin{figure}[H]
    \includegraphics[width=\textwidth,
    height=0.5\textheight]{images/usun_budzet_uzytkownika.png}
    \caption{Diagram sekwencji dla przypadku użycia~\ref{par:userBudgetDelete}~--~Usunięcie budżetu dla określonego użytkownika.
    - scenariusz główny}
\end{figure}

\subsubsection{Opis przypadków użycia -- raporty}

\paragraph{Konfiguracja raportu\newline}
\label{par:reportConfig}
Opis słowny -- istotną funkcją systemu jest generowanie raportów wizualizujących dane zebrane w bazie. Użytkownik powinien mieć możliwość określenia budżetów i zakresu transakcji, których raport ma dotyczyć.

\begin{longtable}{|p{5cm}|p{7cm}|}
 	\hline
	\textbf{Aktor} & Użytkownik \\
	\hline
	\textbf{Warunki początkowe} & Brak \\
	\hline
	\textbf{Opis przebiegu interakcji} & Wybór opcji raportu \\
	\hline
	\textbf{Sytuacje wyjątkowe} & Brak \\
	\hline
	\textbf{Warunki końcowe} & Brak \\
	\hline
\end{longtable}

Wykorzystuje~\ref{par:login}~--~Logowanie do~aplikacji.\\
\indent Funkcja generalizująca dla~\ref{par:reportView} oraz~\ref{par:reportExport}.\\\\
\textit{Scenariusz główny:}
\begin{enumerate}
  \item Użytkownik otwiera stronę ,,Raport''.
  \item Jeśli użytkownik nie jest zalogowany, aplikacja wymaga zalogowania, inicjując~\ref{par:login}~--~Logowanie do~aplikacji.
  \item Aplikacja wyświetla pola~do wprowadzania opcji raportu
  \item Użytkownika wprowadza parametry raportu, takie jak zakres uwzględnionych transakcji (budżet sumaryczny/budżet indywidualny/wszystkie transakcje użytkownika), zakres dat, rodzaj raportu (tabela/wykres).
\end{enumerate}

\paragraph{Wyświetlanie raportu\newline}
\label{par:reportView}

Opis słowny -- użytkownik powinien mieć możliwość wyświetlenia raportu w oknie aplikacji.

\begin{longtable}{|p{5cm}|p{7cm}|}
 	\hline
	\textbf{Aktor} & Użytkownik \\
	\hline
	\textbf{Warunki początkowe} & Wybrane parametry raportu \\
	\hline
	\textbf{Opis przebiegu interakcji} & Wyświetlenie raportu \\
	\hline
	\textbf{Sytuacje wyjątkowe} & Pusty raport \\
	\hline
	\textbf{Warunki końcowe} & Brak \\
	\hline
\end{longtable}

\indent Funkcja specjalizująca dla~\ref{par:reportConfig}~--~Wyświetlanie konfiguracji raportu.\\\\
\textit{Scenariusz główny:}
\begin{enumerate}
  \item[1-4.] Jak w scenariuszu generalizującym~\ref{par:reportConfig}~--~Wyświetlanie konfiguracji raportu.
  \item[5.] Użytkownik wciska przycisk ,,Wyświetl raport''.
  \item[6.] Aplikacja wyświetla okno z raportem w formie zgodnej z konfiguracją.
  \item[7.] Użytkownik wciska przycisk ,,Zamknij''.
  \item[8.] Aplikacja zamyka okno.
\end{enumerate}

\textit{Scenariusz alternatywny~--~brak transakcji w raporcie}
\begin{enumerate}
  \item[1-5.] Jak w scenariuszu głównym.
  \item[6.] Aplikacja wyświetla informację o pustym raporcie.
  \item[7.] Powrót do kroku 1. scenariusza głównego.
\end{enumerate}

\paragraph{Zapisywanie raportu do pliku\newline}
\label{par:reportExport}

Opis słowny -- aplikacja powinna umożliwiać również zapisanie raportu do pliku o określonym formacie.

\begin{longtable}{|p{5cm}|p{7cm}|}
 	\hline
	\textbf{Aktor} & Użytkownik \\
	\hline
	\textbf{Warunki początkowe} & Wybrane parametry raportu \\
	\hline
	\textbf{Opis przebiegu interakcji} & Wybór katalogu i nazwy pliku, zapis raportu \\
	\hline
	\textbf{Sytuacje wyjątkowe} & Pusty raport, przerwanie operacji \\
	\hline
	\textbf{Warunki końcowe} & Raport zapisany do pliku \\
	\hline
\end{longtable}

\indent Funkcja specjalizująca dla~\ref{par:reportConfig}~--~Wyświetlanie konfiguracji raportu.\\\\
\textit{Scenariusz główny:}
\begin{enumerate}
  \item[1-4.] Jak w scenariuszu generalizującym~\ref{par:reportConfig}~--~Wyświetlanie konfiguracji raportu.
  \item[5.] Użytkownik wciska przycisk ,,Zapisz raport''.
  \item[6.] Aplikacja wyświetla okno wyboru katalogu i nazwy pliku.
  \item[7.] Użytkownik wybiera katalog i nazwę pliku i wciska przycisk ,,Zapisz''.
  \item[8.] Aplikacja zapisuje raport w odpowiednim formacie do wybranego pliku i zamyka okno.
\end{enumerate}

\textit{Scenariusz alternatywny~--~brak transakcji w raporcie}
\begin{enumerate}
  \item[1-5.] Jak w scenariuszu głównym.
  \item[6.] Aplikacja wyświetla informację o pustym raporcie.
  \item[7.] Powrót do kroku 1. scenariusza głównego.
\end{enumerate}

\textit{Scenariusz alternatywny~--~przerwanie operacji przez użytkownika}
\begin{enumerate}
  \item[1-6.] Jak w scenariuszu głównym.
  \item[7.] Użytkownik wciska przycisk ,,Anuluj''.
  \item[8.] Aplikacja zamyka okno.
  \item[9.] Powrót do kroku 2. ze scenariusza głównego.
\end{enumerate}

\newpage
\subsection{Wymagania niefunkcjonalne}
% Kombajn
\subsubsection{Bezpieczeństwo}

\paragraph{Ograniczenia dostępu\newline}
Aplikacja nie może pozwalać niezalogowanym użytkownikom na dostęp do danych. Użytkownicy zalogowani mogą mieć dostęp wyłącznie do transakcji własnych oraz transakcji w budżetach zbiorczych. Ponadto użytkownicy mają możliwość edycji wyłącznie własnych transakcji.

\paragraph{Zabezpieczenie przed nieautoryzowanym odczytem danych\newline}
Wszelkie dane przechowywane przez aplikację muszą być zabezpieczone przed odczytem i modyfikacją bez pośrednictwa aplikacji. Dane nie powinny być przechowywane w formacie jawnym.

\paragraph{Transakcyjność\newline}
Aplikacja nie może pozwalać na realizację niekompletnych operacji. W wypadku wystąpienia błędu podczas realizacji operacji dane powinny zostać przywrócone do stanu sprzed rozpoczęcia operacji. Nie powinien być możliwy jednoczesny odczyt i zapis na tych samych danych.

\subsubsection{Użytkowanie}

\paragraph{Wieloinstancyjność\newline}
Aplikacja powinna pozwalać wielu użytkownikom na jednoczesne zalogowanie i korzystanie z jej funkcji. Nie powinna jednak pozwalać na wiele jednoczesnych sesji tego samego użytkownika.

\subsubsection{Wygląd}

\paragraph{Pomoc kontekstowa\newline}
Aplikacja powinna zawierać mechanizm wyświetlania opisów dostępnych opcji wyjaśniających ich przeznaczenie i zastosowanie.
