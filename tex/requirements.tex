\section{Wymagania}
W tej sekcji znajduje się lista wymagań, jakie spełniać powinien budowany system. Podane są~one z~podziałem na~dwie kategorie. Pierwsza to wymagania funkcjonalne -- określające funkcjonalności systemu oraz sposoby ich użycia. Natomiast druga to~wymagania niefunkcjonalne, które opisują ilościowe i~jakościowe warunki działania systemu.

\subsection{Wymagania funkcjonalne}
\subsubsection{Użytkownicy}
Wymagania funkcjonalne dotyczące zarządzaniem użytkownikami w~systemie.
\begin{itemize}
  \item wyświetlanie listy użytkowników,
  \item dodawanie nowego użytkownika,
  \item edycja danych użytkownika,
  \item usunięcie użytkownika.
\end{itemize}

\subsubsection{Konta}
Wymagania funkcjonalne dotyczące zarządzaniem kontami pieniężnymi w~systemie.
\begin{itemize}
  \item wyświetlanie listy kont,
  \item dodawanie nowego konta,
  \item edycja danych konta,
  \item oznaczenie konta jako nieaktywne.
\end{itemize}

\subsubsection{Transakcje}
Wymagania funkcjonalne dotyczące zarządzaniem transakcjami w~systemie.
\begin{itemize}
  \item wyświetlanie listy transakcji,
  \item dodawanie nowej transakcji,
  \item edycja danych transakcji,
  \item usunięcie transakcji.
\end{itemize}
(dodawanie, usuwanie, modyfikacja)

\subsubsection{Budżet}
Wymagania funkcjonalne dotyczące zarządzaniem budżetami w~systemie.\\
% TODO Daniel
(ustalanie limitu miesięcznego na wszystkich użytkowników albo per użytkownik)

\subsubsection{Raporty}
Wymagania funkcjonalne dotyczące generowania raportów w~systemie.
% TODO Kombajn
(generowanie raportów/wykresów)

\subsubsection{Opis przypadków użycia -- użytkownicy}
% TODO Kuba

\subsubsection{Opis przypadków użycia -- konta}
% TODO Maciek

\subsubsection{Opis przypadków użycia -- transakcje}
% TODO Daniel

\subsubsection{Opis przypadków użycia -- budżet}
% TODO Daniel

\subsubsection{Opis przypadków użycia -- raporty}
% TODO Kombajn

\subsection{Wymagania niefunkcjonalne}
% TODO Kombajn
