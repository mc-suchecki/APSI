\section{Wymagania}
W tej sekcji znajduje się lista wymagań, jakie spełniać powinien budowany system. Podane są~one z~podziałem na~dwie kategorie. Pierwsza to wymagania funkcjonalne -- określające funkcjonalności systemu oraz sposoby ich użycia. Natomiast druga to~wymagania niefunkcjonalne, które opisują ilościowe i~jakościowe warunki działania systemu.

\subsection{Wymagania funkcjonalne}
\subsubsection{Użytkownicy}
Wymagania funkcjonalne dotyczące zarządzaniem użytkownikami w~systemie.
\begin{itemize}
  \item wyświetlanie listy użytkowników,
  \item dodawanie nowego użytkownika,
  \item edycja danych użytkownika,
  \item usunięcie użytkownika.
\end{itemize}

\subsubsection{Konta}
Wymagania funkcjonalne dotyczące zarządzaniem kontami pieniężnymi w~systemie.
\begin{itemize}
  \item wyświetlanie listy kont,
  \item dodawanie nowego konta,
  \item edycja danych konta,
  \item oznaczenie konta jako nieaktywne.
\end{itemize}

\subsubsection{Transakcje}
Wymagania funkcjonalne dotyczące zarządzaniem transakcjami w~systemie.
\begin{itemize}
  \item wyświetlanie listy transakcji,
  \item dodawanie nowej transakcji,
  \item edycja danych transakcji,
  \item usunięcie transakcji.
\end{itemize}
(dodawanie, usuwanie, modyfikacja)

\subsubsection{Budżet}
Wymagania funkcjonalne dotyczące zarządzaniem budżetami w~systemie.\\
\begin{itemize}
  \item wyświetlenie budżetów ustawionych w systemie
  \item ustawienie budżetu dla określonego użytkownika
  \item modyfikacja budżetu sumarycznego dla wszystkich użytkowników
  \item modyfikacja budżetu określonego użytkownika
  \item usunięcie budżetu określonego użytkownika
\end{itemize}

\subsubsection{Raporty}
Wymagania funkcjonalne dotyczące generowania raportów w~systemie.
% TODO Kombajn
(generowanie raportów/wykresów)

\subsubsection{Opis przypadków użycia -- użytkownicy}
% TODO Kuba

\subsubsection{Opis przypadków użycia -- konta}
% TODO Maciek

\subsubsection{Opis przypadków użycia -- transakcje}
\textbf{Wyświetlanie listy transakcji}

Główny scenariusz:

\begin{enumerate}
  \item Użytkownik otwiera stronę "Transakcje"
  \item Jeśli użytkownik nie jest zalogowany, aplikacja żąda zalogowania się poprzez (tutaj wypadało by wstawić scenariusz z logowaniem, ale ni ma)
  \item Aplikacja wyświetla listę transakcji powiązanych z kontem danego użytkownika
\end{enumerate}

Scenariusz alternatywny:

\begin{enumerate}
  \item 1 - 3 jak w scenariuszu głównym
  \item Użytkownik wprowadza parametry filtrowania jak nazwa użytkownika, kategoria albo data transakcji
  \item Aplikacja odświeża listę transkacji wyświetlając tylko te, które pasują do podanych filtrów
\end{enumerate}

\textbf{Utworzenie nowej transakcji}

Główny scenariusz:
\begin{enumerate}
  \item 1 - 3 jak w scenariuszu generalizującym "Wyświetlenie listy transakcji"
  \item Użytkownik wciska przycisk "Dodaj transkację"
  \item Aplikacja wyświetla okno z polami do wprowadzania danych dotyczących transakcji
  \item Użytkownik wyświetla rodzaj transakcji (wpływ/wydatek/transfer), konta od/do, wprowadza wartość, datę, opis i klika przycisk "Dodaj"
  \item System weryfikuje wprowadzone dane (np. czy na danym koncie znajduje się wystarczająca ilość pieniędzy)
  \item Jeśli dane są poprawne, system tworzy nową transakcję, przelicza ilość pieniędzy pozostałych na koncie i informuje użytkownika
\end{enumerate}

Scenariusz alternatywny - niepoprawne dane:
\begin{enumerate}
  \item 1 - 7 jak w scenariuszu głównym
  \item System wyświetla informację o wykrytym błędzie i czeka na poprawkę
  \item Użytkownik poprawia dane i wciska przycisk "Dodaj"
  \item Powrót do kroku 7 ze scenariusza głównego
\end{enumerate}

\textbf{Edycja transkacji - scenariusz specjalistyczny dla scenariusza "Wyświetlanie listy transkacji"}

Główny scenariusz:
\begin{enumerate}
  \item 1 - 3 jak w scenariuszu generalizującym "Wyświetlenie listy transakcji"
  \item Użytkownik wybiera jedną z transkacji i wciska przycisk "Edytuj"
  \item Aplikacja wyświetla okno z informacjami o transakcji
  \item Użytkownik modyfikuje dane transakcji (wpływ/wydatek/transfer), konta od/do, wprowadza wartość, datę, opis i klika przycisk "Zapisz"
  \item System weryfikuje wprowadzone dane (np. czy na danym koncie znajduje się wystarczająca ilość pieniędzy)
  \item Jeśli dane są poprawne, system tworzy nową transakcję, przelicza ilość pieniędzy pozostałych na koncie i informuje użytkownika
\end{enumerate}

Scenariusz alternatywny - niepoprawne dane:
\begin{enumerate}
  \item 1 - 7 jak w scenariuszu głównym
  \item System wyświetla informację o wykrytym błędzie i czeka na poprawkę
  \item Użytkownik poprawia dane i wciska przycisk "Zapisz"
  \item Powrót do kroku 7 ze scenariusza głównego
\end{enumerate}

/textbf{Usuwanie transkacji - scenariusz specjalistyczny dla scenariusza "Wyświetlanie listy transakcji"}

Główny scenariusz:

\begin{enumerate}
  \item 1 - 3 jak w scenariuszu generalizującym "Wyświetlenie listy transkacji"
  \item Użytkownik wybiera jedną z transkacji i wciska przycisk "Usuń"
  \item aplikacja wyświetla okienko wymagające potwierdzenia operacji
  \item Użytkownik potwierdza operację wciskając przycisk "Usuń"
  \item Aplikacja zamyka okienko, usuwa transakcję, przelicza ilość pieniędzy na koncie i odświeża listę transakcji.
\end{enumerate}

Scenariusz alternatywny - użytkownik przerywa operację:

\begin{enumerate}
  \item 1 - 5 jak w scenariuszu głównym
  \item Użytkownik wciska przycisk "Anuluj"
  \item Aplikacja zamyka okno
\end{enumerate}

\subsubsection{Opis przypadków użycia -- budżet}

\textbf{Wyświetlenie budżetów ustawionych w systemie}

Główny scenariusz:
\begin{enumerate}
  \item Użytkownik otwiera stronę "Budżety"
  \item Jeśli użytkownik nie jest zalogowany, aplikacja żąda zalogowania się poprzez (tutaj wypadało by wstawić scenariusz z logowaniem, ale ni ma)
  \item Aplikacja wyświetla budżet sumaryczny dla wszystkich użytkowników oraz listę użytkowników z budżetami dla każdego z nich
\end{enumerate}

Scenariusz alternatywny:

\begin{enumerate}
  \item 1 - 3 jak w scenariuszu głównym
  \item Użytkownik wprowadza nazwę konta, którego budżet ma zostać wyświetlony i wciska "Enter"i
  \item Aplikacja odświeża listę budżetów wyświetlając tylko jeden dla podanego użytkownika
\end{enumerate}

\textbf{Ustawienie budżetu dla określonego użytkownika}

Scenariusz główny:
\begin{enumerate}
  \item 1 - 3 jak w scenariuszu generalizującym "Wyświetlnie budżetów ustawionych w systemie"
  \item Użytkownik wciska przycisk "Dodaj"
  \item Aplikacja wyświetla okno z możliwością wyboru konta i wprowadzenia nowego budżetu
  \item Użytkownik wybiera konto, wprowadza nowy budżet i wciska przycisk "Dodaj"
  \item Aplikacja weryfikuje poprawność wprowadzonych danych (np. czy liczba nie jest ujemna)
  \item Jeśli wartość jest poprawna system zapisuje nową wartość budżetu, zamyka okno i aktualizuje stronę z budżetami.
\end{enumerate}

Scenariusz alternatywny - użytkownik przerywa operację:
\begin{enumerate}
  \item 1 - 5 jak w scenariuszu głównym
  \item Użytkownik wciska przycisk "Anuluj"
  \item Aplikacja zamyka okno
\end{enumerate}

Scenariusz alternatywny - wprowadzona wartość jest niepoprawna:
\begin{enumerate}
  \item 1 - 7 jak w scenariuszu głównym
  \item System wyświetla informację o niepoprawnych danych i czeka na ich poprawienie
  \item Użytkownik wprowadza nową wartość i wciska przycisk "Dodaj"
  \item Powrót do kroku 7 ze scenariusza głównego
\end{enumerate}

 /textbf{Modyfikacja budżetu sumarycznego dla wszystkich użytkowników}

Główny scenariusz:
\begin{enumerate}
  \item 1 - 3 jak w scenariuszu generalizującym "Wyświetlnie budżetów ustawionych w systemie"
  \item Użytkownik wciska przycisk "Edytuj" obok pola wyświetlającego budżet sumaryczny
  \item Aplikacja wyświetla okno z możliwością wprowadzenia nowego budżetu sumarycznego
  \item Użytkownik wprowadza nowy budżet i wciska przycisk "Zapisz"
  \item Aplikacja weryfikuje poprawność wprowadzonych danych (np. czy liczba nie jest ujemna)
  \item Jeśli wartość jest poprawna system zapisuje nową wartość budżetu, zamyka okno i aktualizuje stronę z budżetami.
\end{enumerate}

Scenariusz alternatywny - użytkownik przerywa operację:
\begin{enumerate}
  \item 1 - 5 jak w scenariuszu głównym
  \item Użytkownik wciska przycisk "Anuluj"
  \item Aplikacja zamyka okno
\end{enumerate}

Scenariusz alternatywny - wprowadzona wartość jest niepoprawna:
\begin{enumerate}
  \item 1 - 7 jak w scenariuszu głównym
  \item System wyświetla informację o niepoprawnych danych i czeka na ich poprawienie
  \item Użytkownik wprowadza nową wartość i wciska przycisk "Zapisz"
  \item Powrót do kroku 7 ze scenariusza głównego
\end{enumerate}


\textbf{Modyfikacja budżetu dla określonego użytkownika}

Scenariusz główny:
\begin{enumerate}
  \item 1 - 3 jak w scenariuszu generalizującym "Wyświetlnie budżetów ustawionych w systemie"
  \item Użytkownik wciska przycisk "Edytuj" obok pola wyświetlającego budżet dla określonego użytkownika
  \item Aplikacja wyświetla okno z możliwością wprowadzenia nowego budżetu
  \item Użytkownik wprowadza nowy budżet i wciska przycisk "Zapisz"
  \item Aplikacja weryfikuje poprawność wprowadzonych danych (np. czy liczba nie jest ujemna)
  \item Jeśli wartość jest poprawna system zapisuje nową wartość budżetu, zamyka okno i aktualizuje stronę z budżetami.
\end{enumerate}

Scenariusz alternatywny - użytkownik przerywa operację:
\begin{enumerate}
  \item 1 - 5 jak w scenariuszu głównym
  \item Użytkownik wciska przycisk "Anuluj"
  \item Aplikacja zamyka okno
\end{enumerate}

Scenariusz alternatywny - wprowadzona wartość jest niepoprawna:
\begin{enumerate}
  \item 1 - 7 jak w scenariuszu głównym
  \item System wyświetla informację o niepoprawnych danych i czeka na ich poprawienie
  \item Użytkownik wprowadza nową wartość i wciska przycisk "Zapisz"
  \item Powrót do kroku 7 ze scenariusza głównego
\end{enumerate}

\textbf{Usunięcie budżetu dla określonego użytkownika}

Scenariusz główny:
\begin{enumerate}
  \item 1 - 3 jak w scenariuszu generalizującym "Wyświetlnie budżetów ustawionych w systemie"
  \item Użytkownik wciska przycisk "Usuń" obok pola wyświetlającego budżet dla określonego użytkownika
  \item Aplikacja wyświetla okno wymagające potwierdzenia operacji
  \item Użytkownik potwierdza operację wciskając przycisk "Usuń"
  \item System usuwa budżet dla określonego użytkownika, zamyka okno i odświeża stronę z budżetami
\end{enumerate}

Scenariusz alternatywny - użytkownik przerywa operację:
\begin{enumerate}
  \item 1 - 5 jak w scenariuszu głównym
  \item Użytkownik wciska przycisk "Anuluj"
  \item Aplikacja zamyka okno
\end{enumerate}

\subsubsection{Opis przypadków użycia -- raporty}
% TODO Kombajn

\subsection{Wymagania niefunkcjonalne}
% TODO Kombajn
